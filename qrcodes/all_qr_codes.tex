\documentclass[a4paper,oneside]{article}

\usepackage{libertinus}
\usepackage[T1]{fontenc}
\renewcommand\familydefault{\sfdefault} 
\usepackage[margin=2cm]{geometry} % Adjust the margin value as desired
\usepackage{float}
\usepackage{graphicx}
\usepackage{caption}
\captionsetup{justification=centering}
\usepackage{url}
\urlstyle{sf}
\usepackage[hidelinks]{hyperref}
\usepackage{microtype}
\usepackage{parskip}

\newcommand{\myQRCodes}[2]{
        \begin{minipage}[t][7cm][b]{7cm}
        \begin{figure}[H]
            \centering
            \includegraphics[width=0.75\textwidth,keepaspectratio]{#1}
            \caption*{\url{#2}}
        \end{figure}
        \end{minipage}
}

\newcommand{\explanation}{
        \setlength{\skip0}{\parskip}
        \begin{minipage}[t][7cm][c]{7cm}
        \setlength\parskip{\skip0}
Hallo Finder,

Du hältst einen Geocache in der Hand für unsere private Geocache-Rallye.
Bitte lege ihn wieder an seinen Platz zurück, damit unser Spiel funktionieren kann.

Am 30. Juni 2025 werden wir den Cache wieder einsammeln.
Sollte er dich dennoch stören, dann ruf mich bitte an unter der Telefonnummer \href{tel:01637948499}{0163 / 7948499}.

Herzliche Grüße\\
Florian Jackisch
        \end{minipage}
}

\begin{document}

\pagenumbering{gobble}

\myQRCodes{fliegerhalle.png}{https://florian-jackisch.github.io/fliegerhalle}

\myQRCodes{skaterpark.png}{https://florian-jackisch.github.io/skaterpark}
\explanation

\myQRCodes{hbk.png}{https://florian-jackisch.github.io/hbk}
\explanation

\myQRCodes{friedhof.png}{https://florian-jackisch.github.io/friedhof}
\explanation

\myQRCodes{hohetorwall.png}{https://florian-jackisch.github.io/hohetorwall}
\explanation

\myQRCodes{pho_ta.png}{https://florian-jackisch.github.io/pho_ta}
\myQRCodes{soldekk.png}{https://florian-jackisch.github.io/soldekk}

\end{document}